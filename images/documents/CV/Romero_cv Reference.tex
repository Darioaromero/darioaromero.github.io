\documentclass{moderncv}

\usepackage[left=0.8in,top=0.8in,right=0.8in,bottom=0.8in]{geometry}
\moderncvtheme{classic}
\moderncvcolor{grey}  

\usepackage[utf8]{inputenc}
\usepackage{color}

% personal data
\firstname{\huge{Dario Alberto}  \vspace{3mm} \newline }
\familyname{Romero Fonseca}


\homepage{darioaromero.github.io} 
\email{drf312@nyu.edu}
\phone{ (+971) 058 574 4253\\(+1) 202 615 0988 \\}
\address{New York University - Abu Dhabi}{Social Science Division}{Bldg A5-142, UAE}

\newcommand{\up}[1]{\ensuremath{^\textrm{\scriptsize#1}}}

% the ConTeXt symbol
\def\ConTeXt{%
  C%
  \kern-.0333emo%
  \kern-.0333emn%
  \kern-.0667em\TeX%
  \kern-.0333emt}
\definecolor{web}{rgb}{0.2,0.2,0.2}
%\definecolor{web}{rgb}{0.5,0.5,0.5}
%----------------------------------------------------------------------------------
%            content
%----------------------------------------------------------------------------------
\begin{document}
\maketitle
\section{\textbf{Education}}
\cventry{\small 2016--2022}{\small \textbf{Ph.D. Economics}}{Columbia University} {New York, USA}{}{}
\cventry{\small 2011--2012}{\small \textbf{M.A. Economics}}{Universidad de los Andes} {Bogot\'a, Colombia}{}{}
\cventry{\small 2007--2010}{\small \textbf{B.A. Political Science}}{Universidad de los Andes} {Bogot\'a, Colombia}{}{}
\cventry{\small 2006--2010}{\small \textbf{B.A. Economics}}{Universidad de los Andes} {Bogot\'a, Colombia}{}{}

\section{\textbf{Research Interests}}
{Development Economics,  Political Economy,  Economic History and Empirical Methods }

\section{\textbf{Reference}}

\begin{center}
\begin{tabular}{c p{2.cm} c p{2.cm}  c }
\textbf{Juan Fernando Vargas} && \textbf{Kevin Hjortshøj O'Rourke} && \textbf{Jeffrey Jensen} \\
University of Turin && Sciences Po && NYU Abu Dhabi \\
juan.vargas@unito.it && kevin.orourke@sciencespo.fr && jeffrey.jensen@nyu.edu \\

\end{tabular}
\end{center} 








\section{\textbf{Professional Experience}}
\cventry{\small 2022--today}{\small \textbf{New York University -Abu Dhabi}}{Post-Doctoral Associate} {}{}{Abu Dhabi, UAE.}
\cventry{\small 2014--2016}{\small \textbf{Inter-American Development Bank}}{Research Fellow} {}{}{Washington, DC USA.}
\cventry{\small 2013--2014}{\small \textbf{J-PAL LAC}}{Research Analyst} {}{}{Santiago, Chile.}
\cventry{\small 2010--2013}{\small \textbf{Universidad del Rosario}}{Research Assistant} {}{}{Bogot\'a, Colombia.}

\section{\textbf{Publications}}

{Social distancing and COVID-19 under violence: Evidence from Colombia, with Diego Martin (2024) \emph{Journal of Development Economics}, Volume 170. \\}

{Short- and long-run labor market adjustment to import competition, with Juan Blyde, Matias Busso and Kyunglin Park (2023) \emph{Review of International Economics}, pp.1- 18. \\}

{Selective Civilian Targeting: The Unintended Consequences of Partial Peace, with Mounu Prem, Andres Rivera and Juan Vargas (2022) \emph{Quarterly Journal of Political Science}, vol. 7(3), pp.317- 354. \\}

{The Perils of High-Powered Incentives: Evidence from Colombia's False Positives, with Daron Acemoglu, Leopoldo Fergusson, James Robinson and Juan Fernando Vargas (2020). \emph{American Economic Journal: Economic Policy}, vol. 12(3), pp.1- 43. \\ }  

{Improving Access to Preventive Maternal Health Care Using Reminders: Experimental Evidence from Guatemala. With Matias Busso and Dario Salcedo (2017), \emph{Economic Letters}, vol. 161, pp. 43-46.\\} 

{Books or laptops? The effect of shifting from printed to digital delivery of educational content on learning. With Rosangela Bando, Francisco Gallego and Paul J. Gertler (2017). \emph{Economics of Education Review}, vol. 61, pp. 162-173\\} 

{The effects of financial aid and returns information in selective and less selective schools: Experimental evidence from Chile. With Matias Busso, Taryn Dinkelman and Claudia Martínez (2017). \emph{Labour Economics}, Vol. 45, pp. 79-91.\\}

{Insecurity or Perception of Insecurity? Urban Crime and Dissatisfaction with Life: Evidence from the Case of Bogot\'a. (2014). \emph{Peace Economics, Peace Science and Public Policy}, Vol. 20(1), pp. 169-208.}

\section{\textbf{Chapters in Books}}
{Facts and Determinants of Female Labor Supply in Latin America, with Matias Busso. In “Bridging gender gaps? The rise and deceleration of female labor force participation in Latin America”, Leonardo Gasparini and Mariana Marchioni (Eds), 2015.}

\section{\textbf{Working Papers}}
{The Environmental Impact of Civil Conflict: The Deforestation Effect of Paramilitary Expansion in Colombia, with Leopoldo Fergusson and Juan F. Vargas, Cede Working Paper 2014-36.\\}

{An Empire Lost: Spanish Industry and The Effect of Colonial Markets on Peripheral  Innovation (Submitted)}

\section{\textbf{Work in progress}}

{Something Biased This Way Comes: The Effect of Media on House Elections in the US (With Haaris Mateen) [Draft] \\ }

{The (unintended?) effects of US military training during the Cold War in Latin-America (With Diego Martin) [Draft]\\ }

{Andean Winds of Wisdom: Air Pollution and Academic Achievement in Colombia (with Diego Martin and Dario Salcedo) [In progress]\\ }

{Bank Lending and Media Slant (With Haaris Mateen and Elizabeth Berger)  [In progress] \\ }

{Mass Deportations, Economic Networks and Firm Productivity in Guatemala (With Carlos Schmidt-Padilla)  [Data collection] \\ }

\section{\textbf{Teaching Experience}}
\cventry{\small Summer 2021, Fall 2020}{\small Introduction to Econometrics (Undergraduate)}{TA, Columbia University} {}{}{Instructor: Seyhan Erden}
\cventry{\small Spring 2021}{\small Game Theory (Undergraduate)}{TA, Columbia University} {}{}{Instructor: Qingmin Liu}
\cventry{\small Spring 2021}{\small Perspectives on Economic Studies (Ph.D. 1st Year course)}{TA, Columbia University} {}{}{Instructor: Joseph Stiglitz and Suresh Naidu}
\cventry{\small Fall 2019, Fall 2017}{\small Political Economy (Undergraduate)}{TA, Columbia University} {}{}{Instructor: John Marshall}
\cventry{\small Fall 2018}{\small Political Economy (Undergraduate)}{TA, Columbia University} {}{}{Instructor: Alessandra Casella}
\cventry{\small Spring 2018}{\small Principles of Economics  (Undergraduate)}{TA, Columbia University} {}{}{Instructor: Nicola Zaniboni}

\section{\textbf{Conference and Presentations}}
\cvlanguage{\small 2024}{\small RIDGE Santiago, EEA-ESEM Rotterdam, LACEA  Montevideo}{}
\cvlanguage{\small 2023}{\small  RIDGE Montevideo, EHES Vienna, ESOC Washington DC}{}
\cvlanguage{\small 2022}{\small U. de los Andes (Bogot\'a), INSPER (Sao Pablo), U. Javeriana (Bogot\'a) }{}
\cvlanguage{\small 2021}{\small LACEA, Bogot\'a}{}
\cvlanguage{\small 2019}{\small LACEA, Puebla}{}

\section{\textbf{Professional Service}}
\cvcomputer{\textbf{Refereeing}}{}{}{}
{The Quarterly Journal of Economics,  Journal of the European Economic Association,  Oxford Bulletin of Economics and Statistics, Revista Cuadernos de Econom\'ia \\ }


\section{\textbf{Computer skills}}
\cvcomputer{Statistical}{Stata, SPSS, Eviews, R, Matlab, Phyton}{}{}
\cvcomputer{Geographical}{ArcGIS, QGIS, GeoDa}{}{}
\cvcomputer{Word Processor}{MS Office, \LaTeX, Beamer} {}{}

\section{\textbf{Languages}}
\cvcomputer{Spanish}{Native}{English}{Fluent}
\cvcomputer{Portuguese}{Intermediate}{German}{Intermediate}
\cvcomputer{French}{Beginner}{Italian}{Beginner}
\cvcomputer{Arabic}{Basic}{}{}

\pagebreak 
\section{\textbf{Writing Samples}}

{\textbf{The (unintended?) effects of US military training during the Cold War in Latin-America} (With Diego Martin) [Job Market Paper] \\ }
{This paper examines the School of the Americas (SOA), a key program in U.S. foreign policy used to influence Latin America by training Latin American armed forces during the Cold War. We leverage variation among SOA graduates to identify the causal effects of U.S. military training and measure its influence in the region. Our findings show that the SOA program reduced democratic quality and increased government repression. Additionally, we analyze the effects in Argentina and Colombia by exploiting military promotion rules and the distribution of military areas. Zones commanded by SOA graduates experienced higher rates of civilian disappearances during Argentina's military dictatorship and increased civilian victimization during the Colombian civil conflict. These increases occurred without a corresponding rise in military counterinsurgency operations. However, in the long term, we find that SOA promoted democratic values post-Cold War, with cohorts exposed to SOA training showing a rise in support for democracy. This study shows the consequences of foreign military policies on recipient countries, highlighting both the short-term adverse effects on conflict and long-term positive effects on democracy. .}
\vspace*{0.35cm}


{\textbf{Something Biased This Way Comes: The Effect of Media on Local Elections in US} (With Haaris Mateen) [Job Market Paper] \\ }
{Using the staggered expansion of Sinclair Broadcast Group (SBG), a conservative leaning TV station operator, from 2012 to 2017, we study how introducing a biased TV station operator affects electoral outcomes. We use the failed acquisition by SBG of a major station operator to control for the selection effect of market entry. Our findings reveal that SBG acquisition increases the likelihood of a Republican candidate winning House elections, contrasting with a negative impact on Republican performance in presidential elections. Importantly, we document a persistent ideological shift to more conservatism for the winner in House elections, which strengthens over time. When decomposing the ideological effect, we find a shift to relatively more conservatism for \textit{both} Republicans and Democratic candidates in the House elections, even though the pool of Democratic candidates in the primaries becomes more liberal on average. Additionally, we show that Republican candidates receive increased donations in SBG-acquired areas. This study underscores the significance of analyzing electoral settings  beyond national elections where not only voters' preferences but also candidates' strategies and ideology are influenced, highlighting the potential impact of biased media on electoral outcomes and the importance of media ownership regulations.  \\} \vspace*{0.25cm}


{\textbf{An Empire Lost: Spanish Industry and The Effect of Colonial Markets on Peripheral  Innovation (Submitted) \\ }
{This paper examines the impact of international market access on the trajectory of technical change using a historical trade shock that reshaped the Spanish textile industry in the late 19th century. Exploiting the effects of a trade policy change in 1891 that raised out-of-the-empire tariffs and forced the purchase of manufactured cotton goods from the metropole's producers by its colonies, I empirically document a significant increase in cotton textile innovation relative to other fabrics. Moreover, I demonstrate the presence of path dependence in innovation, as the disparity in textile innovation between cotton and other fabrics persisted even after the colonies' independence in 1898. Further analysis reveals that the relative prices of cotton fabrics and benefits accrued by cotton firms played a crucial role in stimulating cotton innovation. These results suggest that the innovation observed was not limited to the mere adoption of foreign technology but instead reflected local conditions in shaping incentives for local innovators to develop technologies tailored to specific local requirements. These findings contribute to the literature on the causal relationship between international trade, foreign markets, and the direction of technical change, shedding light on the possibility of innovation in peripheral countries.}


\end{document}